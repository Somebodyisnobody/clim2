% -*- Mode: LaTeX; Package: CLIM-USER -*-

\chapter {Conventions}
\label {conventions}

This chapter describes the conventions used in this specification and in the
CLIM software itself.


\section {Audience, Goals, and Purpose}

This document, the {\bf CLIM Release 2 Specification}, is intended for vendors.
While it does define the Application Programmer's Interface (API), that is, the
functionality that a customer/consumer would use to write an application, it
also defines the names and functionality of some internal parts of CLIM.  These
``portals'' in implementation space allow one vendor to extend, for example, the
output record mechanism and have it work with another vendor's implementation of
incremental redisplay.  We have attempted to carefully identify the appropriate
``portals'' so that the API can be implemented efficiently, but we have also
tried not to overconstrain the specification so that it restricts creativity of
implementation or the possibility for extension.  This also affects the more
sophisticated application writers who want to go a little below the published
API but still want portable applications.  This document defines which
functionality is part of the advertised API, and which is part of the internal
protocols.

In this document, we refer to three different audiences.  A CLIM \concept{user}
is a person who uses an application program that was written using CLIM.  A CLIM
\concept{programmer} is a person who writes application programs using CLIM.  A
CLIM \concept{implementor} is a programmer who implements CLIM or extends it in
some non-trivial way.


\section {Package Structure}

CLIM defines a variety of packages in order to provide its functionality.  In
general, no symbols except for the symbols in this specification should be added
to those packages.

The \cl{clim-lisp} package is intended to implement as much of the draft X3J13
Common Lisp as possible, independent of the conformance of individual vendors.
(When all Lisp vendors implement X3J13 Common Lisp, the \cl{clim-lisp} package
could be eliminated.)  \cl{clim-lisp} is the version of Common Lisp in which
CLIM is implemented and which the \cl{clim-user} package uses instead of
\cl{common-lisp}.  \cl{clim-lisp} contains only exported symbols, and is locked
in those implementations that allow package locking.

\cl{clim} is the package where the symbols specified in this specification live.
It contains only exported symbols and is locked in those implementations that
allow package locking.

\cl{clim-sys} is the package where useful ``system-like'' functionality lives,
including such things as resources and multi-processing primitives.  It contains
functionality that is not part of Common Lisp, but which is not conceptually the
province of CLIM itself.  It contains only exported symbols and is locked in
those implementations that allow package locking.

No code is written in any of the above packages, but rather code is written for
symbols in the above packages.  None of the above use any other packages (in the
sense of the \cl{:use} option to \cl{defpackage}).  A CLIM implementation might
define a \cl{clim-internals} package that uses each of the above packages, thus
getting the definition of Lisp from \cl{clim-lisp}.  It would then implement the
functionality of the symbols in \cl{clim} and \cl{clim-sys} in the
\cl{clim-internals} package.

\cl{clim-user} is a package that programmers can use if they don't wish to
create their own package.  It is the CLIM analog of \cl{common-lisp-user}.


\section {``Spread'' Point Arguments to Functions \label{spread-vs-point}}

Many functions that take point arguments come in two forms: \concept{structured}
and \concept{spread}.  Functions that take structured point arguments take the
argument as a single \cl{point} object.  Functions that take spread point
arguments take a pair of arguments that correspond to the $x$ and $y$
coordinates of the point.

Functions that take spread point arguments, or return spread point values have
an asterisk in their name, for example, \cl{draw-line*}.


\section {Immutability of Objects}

Most CLIM objects are \concept{immutable}, that is, at the protocol level none
of their components can be modified once the object is created.  Examples of
immutable objects include all of the members of the \cl{region} classes, colors
and opacities, text styles, and line styles.  Since immutable objects by
definition never change, functions in the CLIM API can safely capture immutable
objects without first copying them.  This also allows CLIM to cache immutable
objects.  Constructor functions that return immutable objects are free to either
create and return a new object, or return an already existing object.

A few CLIM objects are \concept{mutable}.  Examples of mutable objects include
streams and output records.  Some components of mutable objects can be modified
once the object has been created, usually via \cl{setf} accessors.

In CLIM, object immutability is maintained at the class level.  Throughout this
specification, the immutability or mutability of a class will be explicitly
specified.

Some immutable classes also allow \concept{interning}.  A class is said to be
interning if it guarantees that two instances that are equivalent will always be
\cl{eq}.  For example, if the class \cl{color} were interning, calling
\cl{make-rgb-color} twice with the same arguments would return \cl{eq} values.
CLIM does not specify that any class is interning, however all immutable classes
are allowed to be interning at the discretion of the implementation.

In some rare cases, CLIM will modify objects that are members of immutable
classes.  Such objects are referred to as being \concept{volatile}.  Extreme
care must be take with volatile objects.  This specification will note whenever
some object that is part of the API is volatile.


\subsection {Behavior of Interfaces}

In this specification, any interfaces that take or return mutable objects can
be classified in a few different ways.

Most functions \concept{do not capture} their mutable input objects, that is,
these functions will either not store the objects at all, or will copy any
mutable objects before storing them, or perhaps store only some of the
components of the objects.  Later modifications to those objects will not
affect the internal state of CLIM.

Some functions \concept{may capture} their mutable input objects.  That is, it
is unspecified as to whether a CLIM implementation will or will not capture the
mutable inputs to some function.  For such functions, programmers should assume
that these objects will be captured and must not modify these objects
capriciously.  Furthermore, it is unspecifed what will happen if these objects
are later modified.

Some programmers might choose to create a mutable subclass of an immutable
class.  If CLIM captures an object that is a member of such a class, it is
unspecified what will happen if the programmer later modifies that object.  If a
programmer passes such an object to a CLIM function that may capture its inputs,
he is responsible for either first copying the object or ensuring that the
object does not change later.

Some functions that return mutable objects are guaranteed to create
\concept{fresh outputs}.  These objects can be modified without affecting the
internal state of CLIM.

Functions that return mutable objects that are not fresh objects fall into two
categories: those that return \concept{read-only state}, and those that return
\concept{read/write state}.  If a function returns read-only state, programmers
must not modify that object; doing so might corrupt the state of CLIM.  If a
function returns read/write state, the modification of that object is part of
CLIM's interface, and programmers are free to modify the object in ways that
``make sense''.


\section {Protocol Classes and Predicates}

CLIM supplies a set of predicates that can be called on an object to determine
whether or not that object satisfies a certain protocol.  These predicates can
be implemented in one of two ways.

The first way is that a class implementing a particular protocol will inherit
from a \concept{protocol class} that corresponds to that protocol.  A protocol
class is an ``abstract'' class with no slots and no methods (except perhaps for
some default methods), and exists only to indicate that some subclass obeys the
protocol.  In the case when a class inherits from a protocol class, the
predicate could be implemented using \cl{typep}.  All of the CLIM region,
design, sheet, and output record classes use this convention.  For example, the
presentation protocol class and predicate could be implemented in this way:

\begin{verbatim}
(defclass presentation () ())

(defun presentationp (object)
  (typep object 'presentation))
\end{verbatim}

Note that in some implementations, it may be more efficient not to use
\cl{typep}, and instead use a generic function for the predicate.  However,
simply implementing a method for the predicate that returns \term{true} is not
necessarily enough to assert that a class supports that protocol; the class must
include the protocol class as a superclass.

CLIM always provides at least one ``standard'' instantiable class that
implements each protocol.

The second way is that a class implementing a particular protocol must simply
implement a method for a predicate generic function that returns \term{true} if
and only if that class supports the protocol (otherwise, it returns
\term{false}).  Most of the CLIM stream classes use this convention.  Protocol
classes are not used in these cases because, as in the case of some of the
stream classes, the underlying Lisp implementation may not be arranged so as to
permit it.  For example, the extended input stream protocol might be implemented
in this way:

\begin{verbatim}
(defgeneric extended-input-stream-p (object))

(defmethod extended-input-stream-p ((object t)) nil)

(defmethod extended-input-stream-p ((object basic-extended-input-protocol)) t)

(defmethod extended-input-stream-p
           ((encapsulating-stream standard-encapsulating-stream))
  (with-slots (stream) encapsulating-stream
    (extended-input-stream-p stream)))
\end{verbatim}

Whenever a class inherits from a protocol class or returns \term{true} from the
protocol predicate, the class must implement methods for all of the generic
functions that make up the protocol.


\section {Specialized Arguments to Generic Functions}

Unless otherwise stated, this specification uses the following convention
for specifying which arguments to generic functions are specialized:

\begin{itemize}
\item If the generic function is a \cl{setf} function, the second argument is
the one that is intended to be specialized.

\item If the generic function is a ``mapping'' function (such as
\cl{map-over-region-set-regions}), the second argument (the object that
specifies what is being mapped over) is the one that is specialized.  The
first argument (the functional argument) is not intended to be specialized.

\item Otherwise, the first argument is the one that is intended to be
specialized.
\end{itemize}


%% Use \tt instead of \cl, such the hairy \cl macro will blow chow
\section {Multiple Value {\tt setf}}

Some functions in CLIM that return multiple values have \cl{setf} functions
associated with them.  For example, \cl{output-record-position} returns the
position of an output record as two values that correspond to the $x$ and $y$
coordinates.  In order to change the position of an output record, the
programmer would like to invoke \cl{(setf~output-record-position)}.  Normally
however, \cl{setf} only takes a single value with which to modify the specified
place.  CLIM provides a ``multiple value'' version of \cl{setf} that allows an
expression that returns multiple values to be used in updating the specified
place.  In this specification, this facility will be referred to as \cl{setf*}.

For example, the modifying function for \cl{output-record-position} might be
called in either of the following two ways:

\begin{verbatim}
(setf (output-record-position record) (values nx ny))

(setf (output-record-position record1) (output-record-position record2))
\end{verbatim}

The second form works because \cl{output-record-position} itself returns two
values.


\section {Sheet, Stream, or Medium Arguments to Macros}

There are many macros that take a sheet, stream, or medium as one of the
arguments, for example, \cl{with-new-output-record} and \cl{formatting-table}.
In CLIM, this argument must be a variable bound to a sheet, stream, or medium;
it may not be an arbitrary form that evaluates to a sheet, stream, or medium.
\cl{t} and sometimes \cl{nil} are usually allowed as special cases; this causes
the variable to be interpreted as a reference to another stream variable
(usually \cl{*standard-output*} for output macros, or \cl{*query-io*} for input
macros).  Note that, while the variable outside the macro form and the variable
inside the body share the same name, they cannot be assumed to be the same
reference.  That is, the macro is free to create a new binding for the variable.
Thus, the following code fragment will not necessarily affect the value of
\arg{stream} outside the \cl{formatting-table} form:

\begin{verbatim}
(formatting-table (stream)
  (setq stream some-other-stream)
  ...)
\end{verbatim}

Furthermore, for the macros that take a sheet, stream, or medium argument, the
position of that variable is always before any forms or other ``inputs''.


\section {Macros that Expand into Calls to Advertised Functions}

Some macros that take a ``body'' argument expand into a call to an advertised
function that takes a functional argument.  This functional argument will
execute the suppled body.  For a macro named ``{\it \cl{with-}environment}'',
the function is generally named ``{\it \cl{invoke-with-}environment}''.  For
example, \cl{with-drawing-options} might be defined as follows:

\begin{verbatim}
(defgeneric invoke-with-drawing-options (medium continuation &key)
  (declare (dynamic-extent continuation)))

(defmacro with-drawing-options ((medium &rest drawing-options) &body body)
  `(flet ((with-drawing-options-body (,medium) ,@body))
     (declare (dynamic-extent #'with-drawing-options-body))
     (invoke-with-drawing-options
       ,medium #'with-drawing-options-body ,@drawing-options)))

(defmethod invoke-with-drawing-options 
           ((medium clx-display-medium) continuation &rest drawing-options)
  (with-drawing-options-merged-into-medium (medium drawing-options)
    (funcall continuation medium)))
\end{verbatim}


\section {Terminology Pertaining to Error Conditions}

When this specification specifies that it ``is an error'' for some situation to
occur, this means that:

\begin{itemize}
\item No valid CLIM program should cause this situation to occur.

\item If this situation does occur, the effects and results are undefined as
far as adherence to the CLIM specification is concerned.

\item CLIM implementations are not required to detect such an error, although
implementations are encouraged to provide such error detection whenever it is
reasonable to do so.
\end{itemize}

When this specification specifies that some argument ``must be a \term{type}''
or uses the phrase ``the \term{type} \arg{argument}'', this means that it is an
error if the argument is not of the specified type.  CLIM implementations are
encouraged, but not required, to generate an argument type error for these
situations.

When this specification says that ``an error is signalled'' in some situation,
this means that:

\begin{itemize}
\item If the situation occurs, an error will be signalled using either
\cl{error} or \cl{cerror}.

\item Valid CLIM programs may rely on the fact that an error will be signalled.

\item Every CLIM implementation is required to detect such an error.
\end{itemize}

When this specification says that ``a condition is signalled'' in some
situation, this is just like ``an error is signalled'' with the exception that
the condition will be signalled using \cl{signal} instead of \cl{error}.
