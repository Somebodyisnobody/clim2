% -*- Mode: LaTeX; Package: CLIM-USER -*-

\chapter {Panes}
\label {panes}

\section {Overview of Panes}

CLIM panes are similar to the gadgets or widgets of other toolkits.  They can be
used by application programmers to compose the top-level user interface of their
applications, as well as auxiliary components such as menus and dialogs.  The
application programmer provides an abstract specification of the pane hierarchy,
which CLIM uses in conjunction with user preferences and other factors to select
a specific ``look and feel'' for the application.  In many environments a CLIM
application can use the facilities of the host window system toolkit via a set
of \concept{adaptive panes}, allowing a portable CLIM application to take on the
look and feel of a native application user interface.

Panes are rectangular objects that are implemented as special sheet classes.  An
application will typically construct a tree of panes that divide up the screen
space allocated to the application frame.  The various CLIM pane types can be
characterized by whether they have children panes or not: panes that can have
other panes as children are called \concept{composite panes}, and those that
don't are called \concept{leaf panes}.  Composite panes are used to provide a
mechanism for spatially organizing (``laying out'') other panes.  Leaf panes
implement gadgets that have some appearance and react to user input by invoking
application code.  Another kind of leaf pane provides an area of the
application's screen real estate that can be used by the application to present
application specific information.  CLIM provides a number of these
\concept{application pane} types that allow the application to use CLIM's
graphics and extended stream facilities.

\concept{Abstract panes} are panes that are defined only in terms of their
programmer interface, or behavior.  The protocol for an abstract pane (that
is, the specified set of initialization options, accessors, and callbacks) is
designed to specify the pane in terms of its overall purpose, rather then in
terms of its specific appearance or particular interactive details.  The
purpose of this abstract definition is to allow multiple implementations of
the abstract pane, each defining its own specific look and feel.  CLIM can
then select the appropriate pane implementation based on factors outside the
control of the application, such as user preferences or the look and feel of
the host operating environment.  A subset of the abstract panes, the
\term{adaptive panes}, have been defined to integrate well across all CLIM
operating platforms.

CLIM provides a general mechanism for automatically selecting the particular
implementation of an abstract pane selected by an application based on the
current frame manager.  The application programmer can override the selection
mechanism by using the name of a specific pane implementation in place of the
abstract pane name when specifying the application frame's layout.  By
convention, the name of the basic, portable implementation of an abstract pane
class can be determined by adding the suffix ``\cl{-pane}'' to the name of the
abstract class.


\section {Basic Pane Construction}

Applications typically define the hierarchy of panes used in their frames using
the \cl{:pane} or \cl{:panes} options of \cl{define-application-frame}.  These
options generate the body of methods on functions that are invoked when the
frame is being adopted into a particular frame manager, so the frame manager can
select the specific implementations of the abstract panes.

There are two basic interfaces to constructing a pane: \cl{make-pane} of an
abstract pane class name, or \cl{make-instance} of a ``concrete'' pane class.
The former approach is generally preferable, since it results in more portable
code.  However, in some cases the programmer may wish to instantiate panes of a
specific class (such as an \cl{hbox-pane} or a \cl{vbox-pane}).  In this case,
using \cl{make-instance} directly circumvents the abstract pane selection
mechanism.  However, the \cl{make-instance} approach requires the application
programmer to know the name of the specific pane implementation class that is
desired, and so is inherently less portable.  By convention, all of the concrete
pane class names, including those of the implementations of abstract pane
protocol specifications, end in ``\cl{-pane}''.

Using \cl{make-pane} instead of \cl{make-instance} invokes the ``look and feel''
realization process to select and construct a pane.  Normally this process is
implemented by the frame manager, but it is possible for other ``realizers'' to
implement this process.  \cl{make-pane} is typically invoked using an abstract
pane class name, which by convention is a symbol in the CLIM package that
doesn't include the ``\cl{-pane}'' suffix.  (This naming convention
distinguishes the names of the abstract pane protocols from the names of classes
that implement them.)  Programmers, however, are allowed to pass any pane class
name to \cl{make-pane}, in which case the frame manager will generally
instantiate that specific class.

\Defprotoclass {pane}

The protocol class that corresponds to a pane, a subclass of \cl{sheet}.  A pane
is a special kind of sheet that implements the pane protocols, including the
layout protocols.
\IfYouWantClass {a} {pane} {pane}

All of the subclasses of \cl{pane} are mutable.

\Defun {panep} {object}

Returns \term{true} if \arg{object} is a \term{pane}, otherwise returns
\term{false}.

\Defun {make-pane} {abstract-class-name \rest initargs}

Selects a class that implements the behavior of the abstract pane
\arg{abstract-class-name} and constructs a pane of that class.  \cl{make-pane}
must be used either within the dynamic scope of a call to
\cl{with-look-and-feel-realization}, or within the \cl{:pane} or \cl{:panes}
options of a \cl{define-application-frame} (which implicitly invokes
\cl{with-look-and-feel-realization}).

\Defgeneric {make-pane-1} {realizer frame abstract-class-name \rest initargs}

The generic function that is invoked by a call to \cl{make-pane}.  The object
that realizes the pane, \arg{realizer}, is established by
\cl{with-look-and-feel-realization}.  Usually, \arg{realizer} is a frame
manager, but it could be another object that implements the pane realization
protocol.  \arg{frame} is the frame for which the pane will be created, and
\arg{abstract-class-name} is the type of pane to create.

\Defmacro {with-look-and-feel-realization} {(realizer frame) \body forms}

Establishes a dynamic context that installs \arg{realizer} as the object
responsible for realizing panes.  All calls to \cl{make-pane} within the context
of \cl{with-look-and-feel-realization} result in \cl{make-pane-1} being invoked
on \arg{realizer}.  This macro can be nested dynamically; inner uses shadow
outer uses.  \arg{body} may have zero or more declarations as its first forms.

Typically \arg{realizer} is a frame manager, but in some cases \arg{realizer}
may be some other object.  For example, within the implementation of pane that
is uses its own subpanes to achieve its functionality, this form might be used
with \arg{realizer} being the pane itself.


\subsection {Pane Initialization Options}

The following options must be accepted by all pane classes.

\defoption {:foreground}
\Defoption {:background}

These options specify the default foreground and background inks for a pane.
Client code should be cautious about passing values for these two options, since
the desktop's look and feel or the user's preferences should usually determine
these values.

\Defoption {:text-style}

This option specifies the default text style that should be used for any sort of
pane that supports text output.  Panes that do not support text output ignore
this option.  Client code should be cautious about passing values for this
option, since the desktop's look and feel or the user's preferences should
usually determine this value.

\Defoption {:name}

This option specifies the name of the pane.  It defaults to \cl{nil}.


\subsection {Pane Properties}

\Defgeneric {pane-frame} {pane}

Returns the frame that ``owns'' the pane.  \cl{pane-frame} can be invoked on any
pane in a frame's pane hierarchy, but it can only be invoked on ``active''
panes, that is, those panes that are currently adopted into the frame's pane
hierarchy.

\Defgeneric {pane-name} {pane}

Returns the name of the pane.

\defgeneric {pane-foreground} {pane}
\Defgeneric {pane-background} {pane}

Return the current foreground and background inks of the pane, respectively.

\issue {SWM} {Why aren't these \cl{medium-foreground} and
\cl{medium-background}, since panes obey the medium protocol to this extent?}


\section {Composite and Layout Panes}

This section describes the various composite and layout panes provided by CLIM,
and the protocol that the layout panes obey.

The layout panes describe in this section are all composite panes that are
responsible for positioning their children according to various layout rules.
Layout panes can be selected in the same way as other panes using \cl{make-pane}
or \cl{make-instance}.  For convenience and readability of application pane
layouts, many of these panes also provide a macro that expands into a
\cl{make-pane} form, passing a list of the panes created in the body of the
macro as the \cl{:contents} argument (described below).  For example, you can
express a layout of a vertical column of two label panes either as:

\begin{verbatim}
(make-instance 'vbox-pane
  :contents (list (make-instance 'label-pane :text "One")
                  (make-instance 'label-pane :text "Two")))
\end{verbatim}

or as:

\begin{verbatim}
(vertically ()
  (make-instance 'label-pane :text "One")
  (make-instance 'label-pane :text "Two"))
\end{verbatim}


\subsection {Layout Pane Options}

\Defoption {:contents}

All of the layout pane classes accept the \cl{:contents} options, which is used
to specify the child panes to be laid out.

\defoption {:width}
\defoption {:max-width}
\defoption {:min-width}
\defoption {:height}
\defoption {:max-height}
\Defoption {:min-height}

These options control the space requirement paramaters for laying out the pane.
The \cl{:width} and \cl{:height} options specify the preferred horizontal and
vertical sizes.  The \cl{:max-width} and \cl{:max-height} options specify the
maximum amount of space that may be consumed by the pane, and give CLIM's pane
layout engine permission to grow the pane beyond the preferred size.  The
\cl{:min-width} and \cl{:min-height} options specify the minimum amount of space
that may be consumed by the pane, and give CLIM's pane layout engine permission
to shrink the pane below the preferred size.

If either of the \cl{:max-width} or \cl{:min-width} options is not supplied, it
defaults to the value of the \cl{:width} option.  If either of the
\cl{:max-height} or \cl{:min-height} options is not supplied, it defaults to the
value of the \cl{:height} option.

\cl{:max-width}, \cl{:min-width}, \cl{:max-height}, and \cl{:min-height} can
also be specified as a relative size by supplying a list of the form
\cl{(\arg{number} :relative)}.  In this case, the number indicates the number
of device units that the pane is willing to stretch or shrink.

The values of these options are specified in the same way as the \cl{:x-spacing}
and \cl{:y-spacing} options to \cl{formatting-table}.  (Note that
\cl{:character} and \cl{:line} may only be used on those panes that display
text, such as a \cl{clim-stream-pane} or a \cl{label-pane}.)

\Defconst {+fill+}

This constant can be used as a value to any of the relative size options.  It
indicates that pane's willingness to adjust an arbitrary amount in the specified
direction.

\defoption {:align-x}
\Defoption {:align-y}

The \cl{:align-x} option is one of \cl{:right}, \cl{:center}, or \cl{:left}.
The \cl{:align-y} option is one of \cl{:top}, \cl{:center}, or \cl{:bottom}.
These are used to specify how child panes are aligned within the parent pane.
These have the same semantics as for \cl{formatting-cell}.

\defoption {:x-spacing}
\defoption {:y-spacing}
\Defoption {:spacing}

These spacing options apply to \cl{hbox-pane}, \cl{vbox-pane}, \cl{table-pane},
and \cl{grid-pane}, and indicate the amount of horizontal and vertical spacing
(respectively) to leave between the items in boxes or rows and columns in table.
The values of these options are specified in the same way as the \cl{:x-spacing}
and \cl{:y-spacing} options to \cl{formatting-table}.  \cl{:spacing} specifies
both the $x$ and $y$ spacing simultaneously.


\subsection {Layout Pane Classes}

\deflpane {hbox-pane}
\Defmacro {horizontally} {(\rest options) \body contents}

The \cl{hbox-pane} class lays out all of its child panes horizontally, from left
to right.  The \cl{horizontally} macro serves as a convenient interface for
creating an \cl{hbox-pane}.

\arg{contents} is one or more forms that are the child panes.  Each form in
\arg{contents} is of the form:

\begin{itemize}
\item A pane.  The pane is inserted at this point and its space requirements are
used to compute the size.

\item A number.  The specified number of device units should be allocated at
this point.

\item The symbol \cl{+fill+}.  This means that an arbitrary amount of space can
be absorbed at this point in the layout.

\item A list whose first element is a number and whose second element evaluates
to a pane.  If the number is less than 1, then it means that that percentage of
excess space or deficit should be allocated to the pane.  If the number is
greater than or equal to 1, then that many device units are allocated to the
pane.  For example:

\begin{verbatim}
(horizontally ()
  (1/3 (make-pane 'label-button-pane))
  (2/3 (make-pane 'label-button-pane)))
\end{verbatim}

would create a horizontal stack of two button panes.  The first button takes
one-third of the space, then second takes two-thirds of the space.
\end{itemize}


\deflpane {vbox-pane}
\Defmacro {vertically} {(\rest options) \body contents}

The \cl{vbox-pane} class lays out all of its child panes vertically, from top to
bottom.  The \cl{vertically} macro serves as a convenient interface for creating
an \cl{vbox-pane}.

\arg{contents} is as for \cl{horizontally}.


\deflpane {hrack-pane}
\Deflpane {vrack-pane}

Similar to the \cl{hbox-pane} and \cl{vbox-pane} classes, except that these
ensure that all children are the same size in the minor dimension.  In other
words, these panes are used to create stacks of same-sized items, such as menu
items.

An \cl{hrack-pane} is created when the \cl{:equalize-height} option to
\cl{horizontally} is \term{true}.  A \cl{vrack-pane} is created when the
\cl{:equalize-width} option to \cl{vertically} is \term{true}.


\deflpane {table-pane}
\Defmacro {tabling} {(\rest options) \body contents}

This pane lays out its child panes in a two-dimensional table arrangement.  Each
of the table is specified by an extra level of list in \arg{contents}.  For example,

\begin{verbatim}
(tabling ()
  (list
    (make-pane 'label :text "Red")
    (make-pane 'label :text "Green")
    (make-pane 'label :text "Blue"))
  (list
    (make-pane 'label :text "Intensity")
    (make-pane 'label :text "Hue")
    (make-pane 'label :text "Saturation")))
\end{verbatim}


\Deflpane {grid-pane}

A \cl{grid-pane} is like a \cl{table-pane}, except that each cell is the same
size in each of the two dimensions.


\deflpane {spacer-pane}
\Defmacro {spacing} {(\rest options) \body contents}

This pane reserves some margin space around a single child pane.  The space
requirement keys that are passed in indicate the requirements for the
surrounding space, not including the requirements of the child.


\deflpane {border-pane}
\Defmacro {bordering} {(\rest options) \body contents}

This layout pane puts a border around its contents.

The \cl{:background} option can be used to control the ink used to draw the
background.


\deflpane {restraining-pane}
\Defmacro {restraining} {(\rest options) \body contents}

Wraps the contents with a pane that prevents changes to the space requirements
for contents from causing relayout of panes outside of the restraining context.
In other words, it prevents the size constraints of the child from propagating
up beyond this point.


\Deflpane {bboard-pane}

A pane that allows its children to be any size and lays them out wherever they
want to be (for example, a desktop manager).


\subsection {Other Composite Pane Classes}

CLIM defines the following additional pane classes, each having at least one
implementation.

\defspane {scroller-pane}
\Defmacro {scrolling} {(\rest options) \body contents}

Creates a composite pane that allows the single child specified by
\arg{contents} to be scrolled.  \arg{options} may include a \cl{:scroll-bar}
option.  The value of this option may be \cl{t} (the default), which indicates
that both horizontal and vertical scroll bars should be created; \cl{:vertical},
which indicates that only a vertical scroll bar should be created; or
\cl{:horizontal}, which indicates that only a horizontal scroll bar should be
created.

The pane created by the \cl{scrolling} will include a \cl{scroller-pane} that
has as children the scroll bars and a \concept{viewport}.  The viewport of a
pane is the area of the window's drawing plane that is currently visible to the
user.  The viewport has as its child the specified contents.

\issue {SWM} {Supply the following four definitions.}

\Defgeneric {pane-viewport} {pane}

\Defgeneric {pane-viewport-region} {pane}

\Defgeneric {pane-scroller} {pane}

\Defgeneric {scroll-extent} {pane x y}


\defspane {label-pane}
\Defmacro {labelled} {(\rest options) \body contents}

Creates a pane that consists of the specified label, which is a string.


\subsection {The Layout Protocol\label{layout-protocol}}

The layout protocol is triggered by \cl{layout-frame}, which is called when a
frame is adopted by a frame manager.

CLIM uses a two pass algorithm to lay out a pane hierarchy.  In the first pass
(called called \concept{space composition}), the top-level pane is asked how
much space it requires.  This may in turn lead to same the question being asked
recursively of all the panes in the hierarchy, with the answers being composed
to produce the top-level pane's answer.  Each pane answers the query by
returning a \concept{space requirement} (or \cl{space-requirement}) object,
which specifies the pane's desired width and height as well as its willingness
to shrink or grow along its width and height.

In the second pass (called \concept{space allocation}), the frame manager
attempts to obtain the required amount of space from the host window system.
The top-level pane is allocated the space that is actually available.  Each
pane, in turn, allocates space recursively to each of its descendants in the
hierarchy according to the pane's rules of composition.

\issue {ILA} {It isn't alway possible to allocate the required space.  What is
the protocol for handling these space allocation failures?  Some kind of error
should be signalled when the constraints can't be satisfied, which can be
handled by the application.  Otherwise the panes will fall where they may.  The
\cl{define-application-frame} macro should provide an option that allows
programmers to conveniently specify a condition handler.}

For many types of panes, the application programmer can indicate the space
requirements of the pane at creation time by using the space requirement options
(described above), as well as by calling the \cl{change-space-requirements}
function (described below).  For example, application panes are used to display
application-specific information, so the application can determine how much
space should normally be given to them.

Other pane types automatically calculate how much space they need based on the
information they need to display.  For example, label panes automatically
calculate their space requirement based on the text they need to display.

A composite pane calculates its space requirement based on the requirements of
its children and its own particular rule for arranging them.  For example, a
pane that arranges its children in a vertical stack would return as its desired
height the sum of the heights of its children.  Note however that a composite is
not required by the layout protocol to respect the space requests of its
children; in fact, composite panes aren't even required to ask their children.

Space requirements are expressed for each of the two dimensions as a preferred
size, a mininum size below which the pane cannot be shrunk, and a maxium size
above which the pane cannot be grown.  (The minimum and maximum sizes can also
be specified as relative amounts.)  All sizes are specified as a real number
indicating the number of device units (such as pixels).

\Defgeneric {change-space-requirements} {pane \key resize-frame \rest space-req-keys} 

This function can be invoked to indicate that the space requirements for
\arg{pane} have changed.  Any of the options that applied to the pane at
creation time can be passed into this function as well.

\arg{resize-frame} determines whether the frame should be resized to accommodate
the new space requirement of the hierarchy.  If \arg{resize-frame} is
\term{true}, then \cl{layout-frame} will be invoked on the frame.  If
\arg{resize-frame} is \term{false}, then the frame may or may not get resized
depending on the pane hierarchy and the \cl{:resize-frame} option that was
supplied to \cl{define-application-frame}.

\Defgeneric {note-space-requirements-changed} {sheet pane}

This function is invoked whenever \arg{pane}'s space requirements have changed.
\arg{sheet} must be the parent of \arg{pane}.  Invoking this function
essentially means that \arg{compose-space} will be reinvoked on \arg{pane}, then
it will reply with a space requirement that is not equal to the reply that was
given on the last call to \arg{compose-space}.

This function is automatically invoked by \cl{change-space-requirements} in the
cases that \cl{layout-frame} isn't invoked.  In the case that \cl{layout-frame}
is invoked, it isn't necessary to call \cl{note-space-requirements-changed}
since a complete re-layout of the frame will be executed.

\Defmacro {changing-space-requirements} {(\key resize-frame) \body body}

This macro supports batching the invocation of the layout protocol by calls
to \cl{change-space-requirements}.  Within the body, all calls to
\cl{change-space-requirements} change the internal structures of the pane and
are recorded.  When the body is exited, the layout protocol is invoked
appropriately.  \arg{body} may have zero or more declarations as its first
forms.


\Defgeneric {compose-space} {pane}

During the space composition pass, a composite pane will typically ask each of
its children how much space it requires by calling \cl{compose-space}.  They
answer by returning \cl{space-requirement} objects.  The composite will then
form its own space requirement by composing the space requirements of its
children according to its own rules for laying out its children.

\Defgeneric {allocate-space} {pane width height}

During the space allocation pass, a composite pane will arrange its children
within the available space and allocate space to them according to their space
requirements and its own composition rules by calling \cl{allocate-space} on
each of the child panes.  \arg{width} and \arg{height} are the width and height
of \arg{pane} in device units.


\Defun {make-space-requirement} {\key (width \cl{0}) (max-width \cl{0}) (min-width \cl{0})
                                      (height \cl{0}) (max-height \cl{0}) (min-height \cl{0})}

Constructs a space requirement object with the given characteristics, \cl{:width},
\cl{:height}, and so on.

\defun {space-requirement-width}             {space-req}
\defun {(setf space-requirement-width)}      {size space-req}
\defun {space-requirement-max-width}         {space-req}
\defun {(setf space-requirement-max-width)}  {size space-req}
\defun {space-requirement-min-width}         {space-req}
\defun {(setf space-requirement-min-width)}  {size space-req}
\defun {space-requirement-height}            {space-req}
\defun {(setf space-requirement-height)}     {size space-req}
\defun {space-requirement-max-height}        {space-req}
\defun {(setf space-requirement-max-height)} {size space-req}
\defun {space-requirement-min-height}        {space-req}
\Defun {(setf space-requirement-min-height)} {size space-req}

These functions read or modify the components of the space requirement
\arg{space-req}.


\section {Application Panes\label{clim-panes}}

In addition to the various layout panes and gadgets, an application usually
needs some space to display application-specific output and receive
application-specific input from the user.  For example, a paint program needs a
``canvas'' pane on which to display the picture and handle the ``mouse
strokes''.  An application frame can use the basic CLIM input and output
services in an application-specific way through use of an \concept{application
pane}.

This section describes the basic CLIM application pane types.  Programmers are
free to customize pane behavior by defining subclasses of these pane classes
writing methods to change the repaint or event-handling behavior.


\subsection {Application Pane Options}

CLIM application frames accept the \cl{:foreground}, \cl{:background},
\cl{:text-style}, and layout pane options.  The space requirement options
(\cl{:width}, \cl{:height}, and so forth) can also take a size specification of
\cl{:compute}, which causes CLIM to run the display function for the pane, and
make the pane large enough to hold the output of the display function.

In addition to the above, CLIM application frames accept the following options:

\Defoption {:display-function}

This is used to specify a function to be called in order to display the contents
of a CLIM stream pane.  CLIM's default top level function,
\cl{default-frame-top-level}, function will invoke the pane's display function
at the appropriate time (see the \cl{:display-time} option).  The value of this
option is either the name of a function to invoke, or a cons whose car is the
name of a function and whose cdr is additional arguments to the function.  The
function will be invoked on the frame, the pane, and the additional function
arguments, if any.  The default for this option is \cl{nil}.

\Defoption {:display-time}

This is used to indicate to CLIM when the pane's display function should be run.
If it is \cl{:command-loop}, CLIM will erase the pane's contents and run the
display function after each time a frame command is executed.  If it is \cl{t},
the pane will be displayed once and not again until \cl{pane-needs-redisplay} is
called on the pane.  If it is \cl{nil}, CLIM will never run the display function
until it is explicitly requested, either via \cl{pane-needs-redisplay} or
\cl{redisplay-frame-pane}.  The default for this option varies.

\Defoption {:incremental-redisplay}

When \term{true}, the redisplay function will initially be executed inside of an
invocation to \cl{updating-output} and the resulting output record will be
saved.  Subsequent calls to \cl{redisplay-frame-pane} will simply use
\cl{redisplay} to redisplay the pane.  The default for this option is \cl{nil}.

\defoption {:text-margin}
\Defoption {:vertical-spacing}

These options specify the default text margin (that is, how much space is left
around the inside edge of the pane) and vertical spacing (that is, how much space
is between each text line) for the pane.  The default for \cl{:text-margin} is
the width of the window, and the default for \cl{:vertical-spacing} is 2.


\defoption {:end-of-line-action}
\Defoption {:end-of-page-action}

These options specify the end-of-line and end-of-page actions to be used for the
pane.  The default for these options are \cl{:wrap} and \cl{:scroll},
respectively.

\Defoption {:output-history}

This option names the class to be used for the output history for the pane.  The
default is \cl{standard-tree-output-history}.


\subsection {Application Pane Classes}

\Defspane {clim-stream-pane}

This class implements a pane that supports the CLIM graphics, extended input and
output, and output recording protocols.  Most application panes will be
subclasses of this class.

\Defspane {interactor-pane}

The pane class that is used to implement ``interactor'' panes.  The default
method for \cl{frame-standard-input} will return the first pane of this type.

For \cl{interactor-pane}, the default for \cl{:display-time} is \cl{nil} and the
default for \cl{:scroll-bars} is \cl{:vertical}.

\Defspane {application-pane}

The pane class that is used to implement ``application'' panes.  The default
method for \cl{frame-standard-output} will return the first pane of this type.

For \cl{application-pane}, the default for \cl{:display-time} is
\cl{:command-loop} and the default for \cl{:scroll-bars} is \cl{t}.

\Defspane {command-menu-pane}

The pane class that is used to implement command menu panes that are not menu
bars.  The default display function for panes of this type is
\cl{display-command-menu}.

For \cl{command-menu-pane}, the default for \cl{:display-time} is
\cl{:command-loop}, the default for \cl{:incremental-redisplay} is \cl{t}, and
the default for \cl{:scroll-bars} is \cl{t}.

\Defspane {title-pane}

The pane class that is used to implement a title pane.  The default display
function for panes of this type is \cl{display-title}.

For \cl{title-pane}, the default for \cl{:display-time} is \cl{t} and the
default for \cl{:scroll-bars} is \cl{nil}.

\Defspane {pointer-documentation-pane}

The pane class that is used to implement the pointer documentation pane.

For \cl{pointer-documentation-pane}, the default for \cl{:display-time} is
\cl{nil} and the default for \cl{:scroll-bars} is \cl{nil}.


\subsection {Making CLIM Application Panes}

Most CLIM application panes will contain more information than can be displayed
in the allocated space, so scroll bars are nearly always desirable.  CLIM
therefore provides a convenient form for creating composite panes that include
the CLIM stream pane, scroll bars, labels, and so forth.

\Defun {make-clim-stream-pane} {\rest options \key type label scroll-bars \allow}

Creates a pane of type \arg{type}, which defaults to \cl{clim-stream-pane}.  If
\arg{label} is supplied, it is a string used to label the pane.
\arg{scroll-bars} may be \cl{t} to indicate that both vertical and horizontal
scroll bars should be included, \cl{:vertical} (the default) to indicate that
vertical scroll bars should be included, or \cl{:horizontal} to indicate that
horizontal scroll bars should be included.

The other options may include all of the valid CLIM application pane options.

\Defun {make-clim-interactor-pane} {\rest options}

Like \cl{make-clim-stream-pane}, except that the type is forced to be
\cl{interactor-pane}.

\Defun {make-clim-application-pane} {\rest options}

Like \cl{make-clim-stream-pane}, except that the type is forced to be
\cl{application-pane}.


\subsection {CLIM Application Pane Functions}

The following functions can be called on any pane that is a subclass of
\cl{clim-stream-pane}.  (Such a pane is often simply referred to as a
\concept{window}.)  These are provided purely as a convenience for programmers.

\Defgeneric {window-clear} {window}

Clears the entire drawing plane by filling it with the background design of the
CLIM stream pane \arg{window}.  If \arg{window} has an output history, that is
cleared as well.  The text cursor position of \arg{window}, if there is one, is
reset to the upper left corner.

\Defgeneric {window-refresh} {window}

Clears the visible part of the drawing plane of the CLIM stream pane
\arg{window}, and then if the window stream is an output recording stream, the
output records in the visible part of the window are replayed.

\Defgeneric {window-viewport} {window}

Returns the viewport region of the CLIM stream pane \arg{window}.  The returned
region will usually be a \cl{standard-bounding-rectangle}.

\Defgeneric {window-erase-viewport} {window}

Clears the visible part of the drawing plane of the CLIM stream pane
\arg{window} by filling it with the background design.

\Defgeneric {window-viewport-position} {window}

Returns two values, the $x$ and $y$ position of the top-left corner of the CLIM
stream pane \arg{window}'s viewport.

\Defgeneric {(setf* window-viewport-position)} {x y window}

Sets the position of the CLIM stream pane \arg{window}'s viewport to \arg{x} and
\arg{y}.


\section {Defining New Pane Types}

This section describes how to define new pane classes.

\subsection {Defining Leaf Panes}

To define a gadget pane implementation, first define the appearance and layout
behavior of the gadget, then define the callbacks, then define the specific user
interactions that trigger the callbacks.

For example, to define an odd new kind of button that displays itself as a
circle, and activates whenever the mouse is moved over it, proceed as follows:

\begin{verbatim}
;; A new kind of button
(defclass sample-button-pane (gadget-pane) ())

;; An arbitrary size parameter
(defparameter *sample-button-radius* 10)

;; Define the sheet's repaint method to draw the button.
(defmethod handle-repaint ((button sample-button-pane) region 
                           &key medium &allow-other-keys)
  (let ((radius *sample-button-radius*)
        (half (round  *sample-button-radius* 2)))
    ;; Larger circle with small one in the center.
    (draw-circle* medium radius radius radius
                  :filled nil)
    (draw-circle* medium radius radius half
                  :filled t)))

;;; Define the pane's compose-space method to always request the
;;; fixed size of the pane.
(defmethod compose-space ((pane sample-button-pane))
  (make-space-requirement :width  (* 2 *sample-button-radius*)
                          :height (* 2 *sample-button-radius*)))
\end{verbatim}

The above code is enough to allow you to instantiate the button pane in an
application frame.  It will fit in with the space composition protocol of, for
example, an \cl{hbox-pane}.  It will display itself as two nested circles.

The next step is to define the callbacks supported by this gadget, and the user
interaction that triggers them.

\begin{verbatim}
;; This default method is defined so that the callback can be invoked
;; on an arbitrary client value without error.
(defmethod value-change-callback
           ((button sample-button-pane) client id value)
  (declare (ignore client id value)))

;; This event processing method defines the rather odd interaction
;; style of this button, to whit: it triggers the activate callback
;; whenever the mouse moves into it.
(defmethod enter-region ((pane sample-button-pane) &key &allow-other-keys)
  (value-change-callback pane (gadget-client pane) (gadget-id pane) nil))
\end{verbatim}


\subsection {Defining Composite Panes} 

\issue {ILA} {To be supplied.}

