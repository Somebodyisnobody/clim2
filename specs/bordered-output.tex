% -*- Mode: LaTeX; Package: CLIM-USER -*-

\chapter {Bordered Output}
\label {bordered-output}

CLIM provides a mechanism for surrounding arbitrary output with some kind of a
border.  The programmer annotates some output-generating code with an advisory
macro that describes the type of border to be drawn.  The following code
produces the output shown in Figure~\ref{border-example}.

For example, the following produces three pieces of output, surrounded by a
rectangular, highlighted with a dropshadow, and underlined, respectively.

\begin{verbatim}
(defun border-test (stream)
  (fresh-line stream)
  (surrounding-output-with-border (stream :shape :rectangle)
    (format stream "This is some output with a rectangular border"))
  (terpri stream) (terpri stream)
  (surrounding-output-with-border (stream :shape :drop-shadow)
    (format stream "This has a drop-shadow under it"))
  (terpri stream) (terpri stream)
  (surrounding-output-with-border (stream :shape :underline)
    (format stream "And this output is underlined")))
\end{verbatim}

\begin{figure}
\ifpsfig\centerline{\psfig{figure=border-example.ps}}\else\vspace{1.50in}\fi
\caption{\label{border-example} Examples of bordered output.}
\end{figure}


\Defmacro {surrounding-output-with-border} {(\optional stream
                                             \key shape (move-cursor \cl{t}))
                                            \body body}

Binds the local environment in such a way the output of \arg{body} will be
surrounded by a border of the specified shape.  Every implementation must
support the shapes \cl{:rectangle} (the default), \cl{:oval}, \cl{:drop-shadow},
and \cl{:underline}.  \cl{:rectangle} draws a rectangle around the bounding
rectangle of the output.  \cl{:oval} draws an oval around the bounding rectangle
of the output.  \cl{:drop-shadow} draws a ``drop shadow'' around the lower right
edge of the bounding rectangle of the output.  \cl{:underline} draws a thin line
along the baseline of all of the text in the output, but does not draw anything
underneath non-textual output.

If the boolean \arg{move-cursor} is \term{true} (the default), then the text
cursor will be moved so that it immediately follows the lower right corner of
the bordered output.

\arg{stream} is an output recording stream to which output will be done.
The \arg{stream} argument is not evaluated, and must be a symbol that is
bound to a stream.  If \arg{stream} is \cl{t} (the default),
\cl{*standard-output*} is used.  \arg{body} may have zero or more
declarations as its first forms.

There are several strategies for implementing borders.  One strategy is to
create a ``border output record'' that contains the output records produced by
the output of \arg{body}, plus one or more output records that represent the
border.  Another strategy might be to arrange to call the border drawer at the
approriate times without explicitly recording it.


\Defmacro {define-border-type} {shape arglist \body body}

Defines a new kind of border named \arg{shape}.  \arg{arglist} must be a subset
of the ``canonical'' arglist below (using \cl{string-equal} to do the
comparison):
\\
\arg{(\key stream record left top right bottom)}
\\

\arg{body} is the code that actually draws the border.  It has lexical access to
\cl{stream}, \cl{record}, \cl{left}, \cl{top}, \cl{right}, and \cl{bottom},
which are respectively, the stream being drawn on, the output record being
surrounded, and the coordinates of the left, top, right, and bottom edges of the
bounding rectangle of the record.  \arg{body} may have zero or more declarations
as its first forms.
